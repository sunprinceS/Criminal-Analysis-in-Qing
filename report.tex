\documentclass{beamer}

\usepackage{fontspec}
\usepackage{xeCJK}
%\setsansfont{Helvetica Neue}
\setCJKmainfont{DFFN_R3.TTC}
%\setCJKfamilyfont{hksy}{Songti TC}
%\setCJKfamilyfont{hkss}{Weibei TC}
%\setCJKfamilyfont{lth}{Lantinghei TC}
\XeTeXlinebreaklocale "zh"
\XeTeXlinebreakskip = 0pt plus 1pt
\linespread{1.3}
\allowdisplaybreaks

\newcommand{\hkss}{\CJKfamily{hkss}}
\newcommand{\hksy}{\CJKfamily{hksy}}
\newcommand{\lth}{\CJKfamily{lth}}
\usepackage{color}
\usepackage{caption}
\usepackage{tikz}
\usepackage{marvosym} % \MVRIGHTarrow
\usepackage{verbatim}
\usepackage{caption}
\usepackage{pgfplots}
\pgfplotsset{
  compat=newest,
  xlabel near ticks,
  ylabel near ticks
}
\usepackage{minted}

\usetheme{EastLansing}
%\usetheme{Copenhagen}
\usetikzlibrary{positioning}
\useinnertheme{rectangles}
\usefonttheme{professionalfonts}

\newcommand{\lw}{0.8mm}
\setbeamercovered{transparent}


\AtBeginSection[]
{
  \begin{frame}<beamer>
    \frametitle{Outline}
    \tableofcontents[currentsection]
  \end{frame}
}

\begin{document}

\title[{數位人文期末專題報告}]
{\LARGE{\lth{清時期的犯罪行為探討}}}
\subtitle{\large{\textcolor[rgb]{0.00,0.50,1.00}{\hksy{Digital Humanity Final Report}}}}
\author[{李旻恆、徐瑞陽、鄭允中、葉}]
{{李旻恆、徐瑞陽、鄭允中、葉}}
\date
{\today}

\begin{frame}
\titlepage
\end{frame}
\begin{frame}
\frametitle{Outline}
\tableofcontents
\end{frame}



\section{竊盜物品觀察}
\subsection{Introduction}
\begin{frame}
	\frametitle{動機}
    我們嘗試利用《淡新檔案》中,財產侵奪的部份(共124案),找出被偷竊及被搶盜的物品,來探討當時的社會風氣及物質生活。
\end{frame}
\subsection{Implementation}
\begin{frame}{Preprocessing}
\begin{block}{常見的pattern}
\begin{itemize}
    \item 單布襪參拾雙、天呢馬掛一件、銅面盆貳個...
    \item 烏布衣、洋綠斜紋襖、紅綢裘...
\end{itemize}
\MVRightarrow{} 需要將這些數字(\textbf{NUMBER})與顏色(\textbf{COLOR})做\textbf{Normalization}
\end{block}
\end{frame}

\begin{frame}{Number Normalization}
\inputminted{python}{code/normalize.py}    
\end{frame}

\begin{frame}{Color Normalization}
不同於數字,我們以為百年前的顏色用法,與現在不同?\\
\MVRightarrow{}利用詞夾子夾出所有的顏色,再仿前面利用正規表示法抓出來擊破!\\\\
\center 黃、紅、青、洋綠、洋紅、黑、紫、白、烏、藍
\end{frame}

\begin{frame}
    有了經過正規化後的文本,詞夾子夾出被竊物品的成效大增!
        \begin{itemize}
        \item 6次iteration
        \item 共232件物品(經由人工確認,並移除子字串,避免重複計算)
        \item 31個詞夾(以量詞為主)
        \begin{itemize}
            \item ⊥...N件
            \item ⊥...N雙
            \item ...N支
            \item 搬搶...⊥
            \item 搶去...⊥
        \end{itemize}
        \end{itemize}
\end{frame}

\begin{frame}
\frametitle{被竊物品初步結果}
\begin{figure}[H]
    	\begin{center}
        	\includegraphics[scale=0.35]{figure/word_cloud.png}
			\caption{被竊物品文字雲}
    	\end{center}
	\end{figure}
\end{frame}

\begin{frame}
    看起來成效不錯,被竊物品主要以牛與金錢為大宗。\\但有許多本質相近的物品卻被分到了不同類。\\
    \MVRightarrow{}``分類``!採用《乾隆地方物品消費與收藏的初步研究》的分類方式
    \begin{itemize}
        \item 各色服飾:褲、衫、裙、襖...
        \item 貴重金屬器皿首飾:銀器、首飾...
        \item 各色鞋襪:元緞靴、羽綾鞋
        \item 牲畜:耕牛、赤牛...
        \item 布料:綢、麻布...
        \item 日常用品:寢具、棉被、燈具...
        \item 穿著飾品或配件:玉手環
        \item 食物:糧食、肉類、藥材
        \item 工具:農具、紡具
        \item 休閑娛樂用具:洋煙、水煙...
        \item 武器:洋銃
    \end{itemize}
\end{frame}

\begin{frame}
    \frametitle{一些有趣的物品}
    \begin{itemize}
        \item 錫滿天光???
        \item 巴參 :「人參菜」,專治咳嗽
        \item 春干:一種很像花枝魷魚的東西,清明拜拜的五味菜之一
        \item 芙蓉膏:消腫的外用中藥
        \item 銀手指:它不是銀製品,是仙人掌
        \item 闊頭:小船
        \item 谷:「穀」
    \end{itemize}
\end{frame}

\subsection{Problem we encountered}

\begin{frame}
\frametitle{遇到的問題}
    利用詞夾子,抓出來的物品不一定為被偷被搶的物品
\end{frame}

\begin{frame}
\frametitle{名稱過於廣義}
\begin{itemize}
    \item 糖、米、牛:\\
    \textit{e.g 管押金長發船運保塹郊金長和諸號``糖````米``等貨}\\
    \MVRightarrow{}處理方式:更\textbf{specify}一些!\\
    因為會被列在偷竊清單上的物品,通常會描述的比較詳細
    \begin{itemize}
        \item 米:振榮米,合順米,錦泉米,振合米,陵茂米
        \item 糖:恆隆糖,姜華舍糖,菁糖
        \item 牛:耕牛,耕赤牛,水牛,赤牛
    \end{itemize}
    \MVRightarrow{}直接使用THDL query,把它們加至物品清單當中,並移掉糖、米、牛等子字串
\end{itemize}
\end{frame}

\begin{frame}
\frametitle{搶劫或偷盜的``工具``}
\begin{itemize}
    \item 小刀:\\
    \textit{e.g} 中途被林李將刀殺傷現在``小刀``一把呈驗
\end{itemize}
檢驗過所有具攻擊性的工具及武器,\\
觀察文本後,手動調出現頻率(斧頭、小刀、洋銃),\\
經此調整後,武器類別便沒有分類了!
\end{frame}

\subsection{Result}
\begin{frame}
    \frametitle{調整後的結果}
    共\textbf{214}類物品,各類分佈如下
        \begin{itemize}
        \item 各色服飾:~$55$
        \item 貴重金屬器皿首飾:~$17$
        \item 各色鞋襪:~$7$
        \item 牲畜:~$5$
        \item 布料:~$9$
        \item 日常用品:~$37$
        \item 穿著飾品或配件:~$13$
        \item 食物:~$33$
        \item 工具:~$21$
        \item 休閑娛樂用具:~$9$
        \item 金錢:~$8$
        \end{itemize}
\end{frame}

\begin{frame}
\frametitle{各種類相關案件出現次數分析}
共$124$案
\begin{figure}[H]
    	\begin{center}
        	\includegraphics[scale=0.35]{figure/cat_bar.png}
			\caption{各種類相關案件次數分析}
    	\end{center}
	\end{figure}
\end{frame}

\subsection{Observation}
\begin{frame}
\frametitle{結論}
OAO
\end{frame}


\section{Task}

\subsection{TOEFL Listening Comprehension}
\begin{frame}
	\frametitle{Task Description}
	\begin{itemize}
	    \item Document-level task
	    \item General task
	    \begin{itemize}
	        \item Overall idea (gist)
	        \item Detail (Factoid QA)
	        \item Inference, drawing conclusion
	    \end{itemize}
	\end{itemize}
	
\end{frame}




\begin{frame}
    \frametitle{Problem we encountered}
    In addition to ASR, we use \textbf{Random Substitution} to test the accuracy, but the performace is too good...
\end{frame}

\section{Future Work}
\begin{frame}{Future Work}
    \begin{block}{Document Preprocessing}
        \begin{itemize}
            \item Compute similarity of the average of GloVe vectors between each sentence and the question.
        \end{itemize}
    \end{block}
\end{frame}

\section{References}
\begin{frame}
	\frametitle{References}
	\begin{enumerate}
		\item 乾隆朝地方物品消費與收藏的初步研究:以四川省巴縣為例(巫仁恕,王大綱)
	\end{enumerate}
\end{frame}

\end{document} 
