\documentclass{beamer}

\usepackage{fontspec}
\usepackage{xeCJK}
%\setsansfont{Helvetica Neue}
\setCJKmainfont{DFFN_R3.TTC}
%\setCJKfamilyfont{hksy}{Songti TC}
%\setCJKfamilyfont{hkss}{Weibei TC}
%\setCJKfamilyfont{lth}{Lantinghei TC}
\XeTeXlinebreaklocale "zh"
\XeTeXlinebreakskip = 0pt plus 1pt
\linespread{1.3}
\allowdisplaybreaks

\newcommand{\hkss}{\CJKfamily{hkss}}
\newcommand{\hksy}{\CJKfamily{hksy}}
\newcommand{\lth}{\CJKfamily{lth}}
\usepackage{caption}
\usepackage{tikz}
\usepackage{marvosym} % \MVRIGHTarrow
\usepackage{verbatim}
\usepackage{caption}
\usepackage{etoolbox}
\usepackage{pgfplots}
\pgfplotsset{
  compat=newest,
  xlabel near ticks,
  ylabel near ticks
}
\usepackage{color}
\definecolor{purple}{RGB}{138,0,184}
\definecolor{white}{RGB}{255,255,255}
\usepackage{minted}
%\usemintedstyle{colorful}
\usetheme{EastLansing}
%\usetheme{Copenhagen}
\usetikzlibrary{positioning}
\useinnertheme{rectangles}
\usefonttheme{professionalfonts}

\newcommand{\lw}{0.8mm}
\setbeamercovered{transparent}


\AtBeginSection[]
{
  \begin{frame}<beamer>
    \frametitle{Outline}
    \tableofcontents[currentsection]
  \end{frame}
}

\begin{document}

\title[{數位人文期末專題報告}]
{\LARGE{\lth{清時期的犯罪行為探討}}}
\subtitle{\large{\textcolor[rgb]{0.00,0.50,1.00}{\hksy{Digital Humanity Final Report}}}}
\author[{李旻恆、徐瑞陽、鄭允中、葉祐銘、魏榆臻}]
{{李旻恆、徐瑞陽、鄭允中、葉祐銘、魏榆臻}}
\date
{\today}

\begin{frame}
\titlepage
\end{frame}
\begin{frame}
\frametitle{Outline}
\tableofcontents
\end{frame}


%%%%deprecated%%%%%%%%%%%%%%%%%%%%%%%%%%%%%%%%%%%%%%%
\iffalse
\section{淡新檔案XML分析}
\subsection{擷取XML檔案中之地名和人名資訊}
\begin{frame}
	\frametitle{擷取XML檔案中之地名和人名資訊}
	\begin{itemize}
	    \item 程式碼實作
	    \begin{itemize}
	        \item 使用Java語言撰寫程式碼,使用dom4j Library,依序對"section" Tag作解析,然後再擷取"LocName"和"PersonName"這些Tag值,再把取得的地名和人名等字串用空格隔開。
	    \end{itemize}
	    \item 擷取地名
	    \begin{itemize}
	        \item 許多Tag僅節錄到一部分的資訊,例如『<LocName>北門外</LocName>保寄寓<LocName>水田庄</LocName>山東昌邑縣屬離城里』,原本於XML中已被切分成許多片段,因此難以組出完整的地名資訊。
	    \end{itemize}
	    \item 擷取人名
	    \begin{itemize}
	        \item 許多Tag僅節錄到一部分的資訊,例如『店主<PersonName>李生</PersonName>侯僱挑工挑綢』,原本於XML中完整人名只被節錄成片段,因此依Tag取得的有些是不完整人名。
	    \end{itemize}
	\end{itemize}
\end{frame}
\fi
%%%%%%%%%%%%%%%%%%%%%%%%%%%%%%%%%%%%%%%%%%%%%%%


%%%%%%%%%徐瑞陽 report %%%%%%%%%%%%%%%%%%%
\section{竊盜物品觀察}
\subsection{Introduction \& Motivation}
\begin{frame}
	\frametitle{動機}
    我們嘗試利用《淡新檔案》中,財產侵奪的部份(共$124$案),找出被偷竊及被搶盜的物品,來探討當時的社會風氣及物質生活。
\end{frame}
\subsection{Implementation}
\begin{frame}{Preprocessing}
\begin{block}{常見的pattern}
\begin{itemize}
    \item 單布襪參拾雙、天呢馬掛一件、銅面盆貳個...
    \item 烏布衣、洋綠斜紋襖、紅綢裘...
\end{itemize}
\MVRightarrow{} 需要將這些數字(\textbf{NUMBER})與顏色(\textbf{COLOR})做\textbf{Normalization}
\end{block}
\end{frame}

\begin{frame}{Number Normalization}
\inputminted{python}{code/normalize.py}    
\end{frame}

\begin{frame}{Color Normalization}
不同於數字,我們以為百年前的顏色用法,與現在不同?\\
\MVRightarrow{}利用詞夾子夾出所有的顏色,再仿前面利用正規表示法抓出來擊破!\\\\
\center \color{yellow}{黃}、\color{red}{紅}、\color{cyan}{青}、\color{green}{洋綠}、\color{magenta}{洋紅}、\color{black}{黑}、\colorbox{black}{\color{white}{白}}、\color{purple}{紫}、\color{black}烏、\color{blue}{藍}
\end{frame}

\begin{frame}
    有了經過正規化後的文本,詞夾子夾出被竊物品的成效大增!
        \begin{itemize}
        \item 6次iteration
        \item 共232件物品(經由人工確認,並移除子字串,避免重複計算)
        \item 31個詞夾(以量詞為主)
        \begin{itemize}
            \item ⊥...N件
            \item ⊥...N雙
            \item ...N支
            \item 搬搶...⊥
            \item 搶去...⊥
        \end{itemize}
        \end{itemize}
\end{frame}

\begin{frame}
\frametitle{被竊物品初步結果}
\begin{figure}[H]
    	\begin{center}
        	\includegraphics[scale=0.35]{figure/word_cloud.png}
			\caption{被竊物品文字雲}
    	\end{center}
	\end{figure}
\end{frame}

\begin{frame}
    看起來成效不錯,被竊物品主要以金錢、牛、衣服為大宗。\\但有許多本質相近的物品卻被分到了不同類。\\
    \MVRightarrow{}``分類``!採用《乾隆地方物品消費與收藏的初步研究》的分類方式
    \begin{itemize}
        \item 各色服飾:褲、衫、裙、襖...
        \item 貴重金屬器皿首飾:銀器、首飾...
        \item 各色鞋襪:元緞靴、羽綾鞋
        \item 牲畜:耕牛、赤牛...
        \item 布料:綢、麻布...
        \item 日常用品:寢具、棉被、燈具...
        \item 穿著飾品或配件:玉手環
        \item 食物:糧食、肉類、藥材
        \item 工具:農具、紡具
        \item 休閑娛樂用具:洋煙、水煙...
        \item 武器:洋銃
    \end{itemize}
\end{frame}

\begin{frame}
    \frametitle{一些有趣的物品}
    \begin{itemize}
        \item 錫滿天光???
        \item 巴參 :「人參菜」,專治咳嗽
        \item 春干:一種很像花枝魷魚的東西,清明拜拜的五味菜之一
        \item 芙蓉膏:消腫的外用中藥
        \item 銀手指:它不是銀製品,是仙人掌
        \item 闊頭:小船
        \item 谷:「穀」
    \end{itemize}
\end{frame}

\begin{frame}
\frametitle{遇到的問題}
    利用詞夾子,抓出來的物品``不一定``為被偷被搶的物品
\end{frame}

\begin{frame}
\frametitle{名稱過於廣義}
\begin{itemize}
    \item 糖、米、牛:\\
    \textit{e.g 管押金長發船運保塹郊金長和諸號``糖````米``等貨}\\
    \MVRightarrow{}處理方式:更\textbf{specify}一些!\\
    因為會被列在偷竊清單上的物品,通常會描述的比較詳細
    \begin{itemize}
        \item 米:振榮米,合順米,錦泉米,振合米,陵茂米
        \item 糖:恆隆糖,姜華舍糖,菁糖
        \item 牛:耕牛,耕赤牛,水牛,赤牛
    \end{itemize}
    \MVRightarrow{}直接使用THDL query,把它們加至物品清單當中,並移掉糖、米、牛等子字串
\end{itemize}
\end{frame}

\begin{frame}
\frametitle{搶劫或偷盜的``工具``}
檢驗過所有具攻擊性的工具及武器,\\
觀察文本後,手動調出現頻率(斧頭、小刀、洋銃),\\
經此調整後,武器類別便沒有分類了!
\begin{itemize}
    \item 小刀:\\
    \textit{e.g} 中途被林李將刀殺傷現在``小刀``一把呈驗
\end{itemize}
\end{frame}

\subsection{Result \& Conclusion}
\begin{frame}
    \frametitle{調整後的結果}
    共\textbf{214}類物品,各類分佈如下
        \begin{itemize}
        \item 各色服飾:~$55$
        \item 貴重金屬器皿首飾:~$17$
        \item 各色鞋襪:~$7$
        \item 牲畜:~$5$
        \item 布料:~$9$
        \item 日常用品:~$37$
        \item 穿著飾品或配件:~$13$
        \item 食物:~$33$
        \item 工具:~$21$
        \item 休閑娛樂用具:~$9$
        \item 金錢:~$8$
        \end{itemize}
\end{frame}

\begin{frame}
\frametitle{各種類相關案件出現次數分析}
共$124$案
\begin{figure}[H]
    	\begin{center}
        	\includegraphics[scale=0.35]{figure/cat_bar.png}
			\caption{各種類相關案件次數分析}
    	\end{center}
	\end{figure}
\end{frame}
\begin{frame}
\frametitle{結論}
OAO
\end{frame}
%%%%%%%%%%%%%%%%%%%%%%%%%%%%%%%%%

%%%%%    鄭允中report  %%%%%%
\section{淡新檔案文本重分類之探討}
\subsection{Introduction \& Motivation}
\begin{frame}
	\frametitle{動機:分類與詮釋}
	脈絡拆解
	\begin{itemize}
	\item 淡新檔案是由西方的分類觀點來作為淡新檔案的分類標準。
	\item 西方的分類法拆解掉了當時的脈絡。
	\item 西方主要以民事、刑事、行政作為分類依據
	\end{itemize}
	重建脈絡
	\begin{itemize}
	\item 以自動化的方法重建當時的分類
	\item 當時中國通常以六部吏、戶、禮、兵、刑、工作為分類標準。
	\end{itemize}
\end{frame}

\subsection{Chinese Documents Preprocessing}
\begin{frame}
	\frametitle{XML file parsing}
	XML設計用來傳送及攜帶資料資訊,不用來表現或展示資料。XML可以允許使用者自行定義所需的標籤(tags),並任意啟動定義、轉換、驗證等工作,同時可在網頁和應用程式間直接讀取及傳遞資料。
	    \begin{itemize}
	        \item 豐富檔案(Rich Documents)- 自定檔案描述並使其更豐富
	        \item 後設資料(Metadata)- 描述其它檔案或網路資訊
	        \item 配置文件(Configuration Files)- 描述軟體設定的參數
	    \end{itemize}
\end{frame}

\begin{frame}
\frametitle{XML file parsing}
\inputminted{r}{code/xml_parsing.r}
\end{frame}

\begin{frame}
    \frametitle{Segmentation}
        \textit{Rwordseg} 是一個R環境下的中文分詞工具,\\使用 \textit{rJava} 調用 Java 分詞工具 \texit{Ansj},\\採用隱式馬可夫模型(Hidden Markov Model, HMM)。\\但中文表意的字符之間沒有空格連接,因此需要中文分詞工具的協助。
\end{frame}

\begin{frame}
    \frametitle{去除停用字}
    停用詞出現頻繁,但它們跟文章的內容近乎無關,\\
    造成檢索分類等方法的雜訊問題。\\
    e.g 之、以、何...等等。
    \begin{itemize}
        \item 經驗法則:\\$30$ most common words account for $30\%$ of the tokens in written text
        \item Zipf's law : rank × freq = constant
    \end{itemize}
\end{frame}

\begin{frame}
    \frametitle{Term Document Matrix}
    最後產生文本探勘最常使用的 Term Document Matrix。
    \begin{figure}[H]
    	\begin{center}
        	\includegraphics[scale=0.35]{figure/tdm.png}
			\caption{Term Document Matrix}
    	\end{center}
	\end{figure}
\end{frame}

\subsection{Language Model Implementation}
\begin{frame}
    \frametitle{Language Model}
        Generate a piece of text by generating each word. (using unigram model)
        \begin{figure}[H]
    	\begin{center}
        	\includegraphics[scale=0.35]{figure/lm.png}
			\caption{Language Model}
    	\end{center}
	\end{figure}
\end{frame}
\begin{frame}
    \frametitle{Smoothing}
    Assign a low probability to words not observed in the training corpus.\\
Connecting with TF-IDF weighting and document length normalization.
 \begin{figure}[H]
    	\begin{center}
        	\includegraphics[scale=0.30]{figure/math01.png}
    	\end{center}
	\end{figure}
	 \begin{figure}[H]
    	\begin{center}
        	\includegraphics[scale=0.28]{figure/math02.png}
    	\end{center}
	\end{figure}
\end{frame}

\begin{frame}
    \frametitle{Naive Bayes Classifier : word probability}
    意義為在某一個主題下某字出現的機率。這是有smoothing的版本,\\多了1和|V|兩個參數可調節 smoothing 的影響力。
    \begin{figure}[H]
    	\begin{center}
        	\includegraphics[scale=0.34]{figure/math03.png}
    	\end{center}
	\end{figure}

\end{frame}

\begin{frame}
    \frametitle{Naive Bayes Classifier : The class prior probabilities}
    意義為在某一個主題出現的機率。這也是有smoothing的版本,\\多了1和|C|兩個參數可調節 smoothing 的影響力。
    \begin{figure}[H]
    	\begin{center}
        	\includegraphics[scale=0.40]{figure/math04.png}
    	\end{center}
	\end{figure}

\end{frame}

\begin{frame}
    \frametitle{Using Naive Bayes Classifier}
    算出某主題出現在該文章的機率,\\再用貝氏定理反轉成該文章是某主題的機率。
    \begin{itemize}
        \item 分母不用算,因為每一個要比較的機率值除的分母都一樣
        \item 分子取log相加,不然會有超出精度的問題。
    \end{itemize}
    \begin{figure}[H]
    	\begin{center}
        	\includegraphics[scale=0.33]{figure/math05.png}
    	\end{center}
	\end{figure}
\end{frame}

\subsection{Result \& Conclusion}
\begin{frame}
    \frametitle{Result}
    藉由建立古代六部分類的分類器(在此簡單以刑事為例),\\
    可以將西方分類法中民事、行政的案件放入分類器,\\即可篩選出可疑的檔案做人工精確的分類定位。
    \begin{figure}[H]
    	\begin{center}
        	\includegraphics[scale=0.35]{figure/result.png}
			\caption{Result}
    	\end{center}
	\end{figure}
\end{frame}
\begin{frame}
    \frametitle{結論}
    藉由中國與西方案件分類重定位,釐清兩者間刑事案件的分野,以重新釐清不同文化對於刑事事件的詮釋。
\end{frame}





\section{References}
\begin{frame}
	\frametitle{References}
	\begin{enumerate}
		\item 乾隆朝地方物品消費與收藏的初步研究:以四川省巴縣為例(巫仁恕,王大綱)
	\end{enumerate}
\end{frame}

\end{document} 
